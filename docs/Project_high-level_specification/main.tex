\documentclass{article}
\usepackage{graphicx} % Required for inserting images
\usepackage{geometry} % Required for setting margins

\geometry{
	a4paper,
	total={170mm,242mm},
	left=20mm,right=20mm,
	top=20mm,bottom=25mm
}

\title{Prenotalo: project high-level specification}
\author{Cecchetto A., d'Antonio S., Perrella C., Roccia E., Zerpa Ruiz J. E.}
\date{January 2024}

\begin{document}

\maketitle
% Required contents:
% 1. Objectives of the project
% 2. What the system should do
% 3. What are the potential users
% 4. Most important use cases

\section{Introduction}
% Temporary introduction, needs more work.
In this document, we provide the initial idea and high-level specification of the project describing what are its objectives, what the system should do, what are the potential users and the more important use cases. The project deals with the development of a distributed system, made up of different micro-services, to manage bookings and related operations. Prenotalo is at its core a simple but comprehensive booking system.
\begin{table}[!h]
\centering
\begin{tabular}{l l}
\multicolumn{2}{l}{\textbf{Project Glossary}} \\
\hline
\textbf{Term} & \textbf{Definition} \\
\hline
Consumer & Someone that requests or books a service offered by a Producer \\
\hline
Producer & Someone that provides services to Consumers \\
\hline
User & Platform user, so both Consumer and Producer \\
\hline
\end{tabular}

\end{table}

% Maybe extend additional services and utilities.
The aim of this project is to create a distributed system, made up of different
micro-services, that primarily provides booking management functionalities
and also integrating additional services and utilities.

\section{Objectives}
% Maybe extend this section.
With objectives of the project we mean what the system aims to facilitate or
solve. The system, Prenotalo, provides booking management functionalities also
integrating additional services and utilities, like notification and analytics.

It facilitates interaction between consumers and producers. It aims to provide
a simple and straightforward way for consumers to find, book, and manage their
appointments or reservations. It helps producers manage their resources
effectively.

\section{Functionalities}
For functionalities, we mean what the application should do in order to satisfy
concretely the objectives. The main functionalities of the system will be:
\begin{itemize}
    \item Provide a way to register user data.
    % Does it work like a concert (so there is an event and you buy a ticker
    % for the event) or like a doctor office (so there is the doctor
    % and you request an appointment). An event is like a service (time
    % slots, hair cut,...)
    \item Provide a way for a producer to create new events.
    \item Provide a way for a consumer to look for available events.
    \item Provide a way for a consumer to book an event.
    \item Provide a way for a consumer to look for booked events.
    \item Provide a way for a consumer to cancel a booking.
    \item Provide a way for a producer to cancel events.
    \item Provide a way for a consumer to pay for an attended event.
    \item Send consumers notifications about booked events, e.g. reminders.
    \item Provide a way for a producer to create notifications about an event,
	e.g. broadcast messages. 
    \item Collect surveys about completed events.
    \item Provide data analytics obtained by the system to producer.
    \item Provide a web application interface to use the system.
    \item Provide a mobile application interface to use the system.
\end{itemize}

\section{Potential users}
% Good. Satisfies 3
Such a system could potentially be used by any small business, merchant or
self-employed practitioner. This booking system aims to be general enough to be
useful any time there is a service offered in slots of time. Hence, some
examples of potential users might be lawyers, doctors, hair salons, schools
like driving schools, dancing schools, etc. On the other hand, clients of such
businesses might use the system to book the offered services.

\section{Main use cases}

The most relevant use cases of the system are the following:

% What is an event? what is a service?
\begin{itemize}
	% Should we also include that a producer publishes services?
	\item \textbf{A producer creates an event}: a producer defines a name
		for the event, specifying its duration and a start time. Then
		the event is published and so is viewable by consumers.
	\item \textbf{A consumer requests an event}: a consumer finds the
		appropriate provider that offers the needed services. Then
		the consumer selects the service from the list of offered services,
		of the relative producer. Then selects a time slot and sends a 
		request for an event to the producer. Consequently, the
		producer can either accept or discard the request recieved.
	% How a consumer can find events created by a producer?
	\item \textbf{A consumer books an event}: a consumer searchs for a producer,
		cheking if there are events to which he can partecipate.
		The consumer then can sent a booking request for the event of interest.
		The producer the recieves a request of partecipation that can either be
		accepted or discarded.
	\item \textbf{A consumer pays for a booked event}: a consumer checks for its
		booked events. Selects an event that supports payment trough the system.
		The payment sequence is then managed by an external service that then
		notifies wheter the payment was succesful or not.
\end{itemize}


\end{document}
