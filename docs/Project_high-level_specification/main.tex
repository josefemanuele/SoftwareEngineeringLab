\documentclass{article}
\usepackage{graphicx} % Required for inserting images
\usepackage{geometry} % Required for setting margins

\geometry{
	a4paper,
	total={170mm,242mm},
	left=20mm,right=20mm,
	top=20mm,bottom=25mm
}

\title{Prenotalo: project high-level specification}
\author{Cecchetto A., d'Antonio S., Perrella C., Roccia E., Zerpa Ruiz J. E.}
\date{January 2024}

\begin{document}

\maketitle

\begin{table}[!h]
\centering
\begin{tabular}{l l}
\multicolumn{2}{l}{\textbf{Glossary}} \\
\hline
\textbf{Term} & \textbf{Definition} \\
\hline
Consumer & Someone that requests or books a service offered by a Producer \\
\hline
Producer & Someone that provides services to Consumers \\
\hline
User & Platform user, so both Consumer and Producer \\
\hline
Event & A defined time-slot in which a Producer concretely provides a service
	to one or multiple Consumers\\
\hline
Service & A service that can be offered by a Producer to a Consumer in a
	time-slotted manner\\ \hline
\end{tabular}

\end{table}

% Required contents:
% 1. Objectives of the project
% 2. What the system should do
% 3. What are the potential users
% 4. Most important use cases

\section{Introduction}
% Temporary introduction, needs more work.
In this document, we provide the initial idea and high-level specification of
the project describing what are its objectives, what are its core
functionalities, what are the potential users and the most important use cases.

The aim of this project is to create a distributed system, made up of different
micro-services. 

\section{Objectives}
% Maybe extend this section.
With objectives of the project we mean what the system aims to facilitate or
solve. The system "Prenotalo" provides booking management functionalities,
integrating also additional services and utilities like notification and analytics.

It facilitates interaction between consumers and producers. It aims to provide
a simple and straightforward way for consumers to find, book, and manage their
appointments or reservations. It helps producers manage their resources
effectively.

\section{Functionalities}

For clarity sake we define as \emph{producer} an user that provides some
kind of services that can be offered in a time-slotted manner. Instead the
\emph{consumer} is the one that requests or uses these services.

For functionalities, we mean what the application should do in order to satisfy
concretely the objectives. The functionalities of the system will be:
\begin{itemize}
    \item Provide a way to register user data.
    \item Provide a way for a producer to declare offered services
    \item Provide a way for a producer to create new events.
    \item Provide a way for a producer to cancel events.
    \item Provide a way for a consumer to look for available events.
    \item Provide a way for a consumer to book an event.
    \item Provide a way for a consumer to look for booked events.
    \item Provide a way for a consumer to cancel a booking.
    \item Provide a way for a consumer to pay for an attended event.
    \item Send consumers notifications about booked events, e.g. reminders.
    \item Provide a way for a producer to create notifications about an event,
	e.g. broadcast messages. 
    \item Collect surveys about completed events.
    \item Provide data analytics obtained by the system to producer.
    \item Provide a web application interface to use the system.
    \item Provide a mobile application interface to use the system.
\end{itemize}
Follows a more detailed and lenghty description of the core functionalities
of the system:

\begin{itemize}

	\item Producers can declare and show to consumers what services they
		offer and optionally their price. Example: consider as a
		producer a dance school. They
		can offer as services individual hip-hop dance classes and salsa
		classes, both for a duration of one hour.

	\item A consumer can ask to book a service offered by a producer,
		resulting in the creation of an event. Example: let's consider
		a consumer for the dance school aforementioned. He wants to book
		an individual salsa class for the afternoon. He selects the
		salsa class and then chooses from the time availabilities. In
		this way he sends a request to the producer to create a
		dedicated event for him.

	\item A producer can create "public" events, to which any consumer can
		partecipate. Example: the dance school offers a group course of
		hip-hop dance every saturday afternoon to 18:00 to 20:00. The
		dance school can create a public event for this group classes.
		Consumers can see these public events and send a request to
		partecipate.

	\item Events can be cancelled by producers. Consumers can also cancel
		their booking to events. The cancellation of an event triggers
		a notification that is sent to the other party.

	\item Producers can also send notifications to consumers that
		partecipate to an event. Example: a producer can send a
		notification to all those that partecipate to a public event,
		the same happens for events that are created after a consumer
		request.
\end{itemize}

\section{Potential users}
This booking system aims to be general enough to be
useful any time there is a service offered in slots of time. Hence, some
examples of potential users might be lawyers, doctors, any self-employed
practitioner or small buisnesses like hair salons, schools like driving
schools, dancing schools, etc. Naturally, also clients of such businesses are
potential users of the system since it would be used to book the services
offered by them.

\section{Main use cases}

We decided to not follow strictly the UML structure of the use cases both for ours
easiness and of the reader. However it is included something that resembles the
main flow of them.

The most relevant use cases of the system are the following:

% Sorry J. i know that we discussed that the system is not marketplace-like but
% i still don't understand it. So these use cases unfortunately resembles that
% kind of behaviour.
\begin{itemize}
	\item \textbf{A producer defines its offered services}: a producer can
		declare to customers what set of services they offer. When creating
		a new service the producer must define its name and a duration in
		hours. Optionally the price for that service can be declared.
	\item \textbf{A producer creates an event}: a producer defines a name
		for the event, specifying its duration and a start time. 
		Optionally it can be declared a partecipation cost. Then the
		event is published and so is viewable by consumers.
	\item \textbf{A consumer requests an event}: a consumer finds the
		appropriate provider that offers the needed services. Then
		the consumer selects the service from the list of offered services,
		of the relative producer. Then selects a time slot and sends a 
		request for an event to the producer. The producer is then notified
		and can consequently accept or discard the request.
	\item \textbf{A consumer books an event}: a consumer searchs for a producer,
		cheking if there are events to which he can partecipate.
		The consumer then selects the event of interest and sends a
		booking request. The producer then recieves a request of
		partecipation that can either be accepted or discarded.
	\item \textbf{A consumer pays for a booked event}: a consumer checks for events
		that he has booked. Selects an event that supports payment
		trough the system. The payment sequence is then managed by an
		external service that then notifies wheter the payment was
		succesful or not.
	% Maybe we shoud add an use case for sending notifications
\end{itemize}

\end{document}
