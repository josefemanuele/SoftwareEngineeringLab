\documentclass{article}
\usepackage{graphicx} % Required for inserting images
\usepackage{geometry} % Required for setting margins

\geometry{
	a4paper,
	total={170mm,242mm},
	left=20mm,right=20mm,
	top=20mm,bottom=25mm
}

\title{Prenotalo: project high-level specification.}
\author{Cecchetto A., d'Antonio S., Perrella C., Roccia E., Zerpa Ruiz J. E.}
\date{January 2024}

\begin{document}

\maketitle
% Required contents:
% 1. Objectives of the project
% 2. What the system should do
% 3. What are the potential users
% 4. Most important use cases

% Temporary introduction, needs more work.
In this document, we provide the initial idea and high-level specification of the project describing what are its objectives, what the system should do, what are the potential users and the more important use cases. The project deals with the development of a distributed system, made up of different micro-services, to manage bookings and related operations. Prenotalo is at its core a simple but comprehensive booking system.

% Maybe extend additional services and utilities.
The aim of this project is to create a distributed system, made up of different
micro-services, that primarily provides booking management functionalities
and also integrating additional services and utilities, like notification and analytics.

\section{Objectives}
% Maybe extend this section.
With objectives of the project we mean what the system aims to facilitate or solve. The system, Prenotalo, provides booking management functionalities also integrating additional services and utilities.

It facilitates interaction between clients and service providers. It aims to provide a simple and straightforward way for clients to find, book, and manage their appointments or reservations. It helps service providers manage their resources effectively.

\section{Functionalities}
For functionalities, we mean what the application should do in order to satisfy concretely the objectives. For the sake of clarity, we define as client the person that makes reservations or books a service. With the term user we mean platform user, so it refers both to clients and service providers. The main functionalities of the system will be:
\begin{itemize}
    \item Provide a way to register user data.
    % Does it work like a concert (so there is an event and you buy a ticker for the event) or like a doctor office (so there is the doctor and you request an appointment).
    \item Provide a way for a service provider to create new events.
    \item Provide a way for a client to look for available events.
    \item Provide a way for a client to book an event.
    \item Provide a way for a client to look for booked events.
    \item Provide a way for a client to cancel a booking.
    \item Provide a way for a service provider to cancel events.
    \item Provide a way for a client to pay for an attended event.
    \item Send clients notifications about booked events, e.g. reminders.
    \item Provide a way for a service provider to create notifications about an event, e.g. broadcast messages.
    \item Collect surveys about completed events.
    \item Provide data analytics obtained by the system to service providers.
    \item Provide a web application interface to use the system.
    \item Provide a mobile application interface to use the system.
\end{itemize}

\section{Potential users}
% Good. Satisfies 3
Such a system could potentially be used by any small business, merchant or
self-employed practitioner. This booking system aims to be general enough to be
useful any time there is a service offered in slots of time. Hence, some 
examples of potential users might be lawyers, doctors, hair salons, schools like driving schools, dancing schools, etc. On the other hand, clients of such businesses might use the system to book the offered services.

\section{Sample use cases}
% I would rewrite this section, it should be way more descriptive IMO.
% It should include someone that books something, someone that offers a
% service, and so on... Check what i have written after the original text
To have in mind some actual use cases, you might think of lawyer offering
hers/his services by appointment, providing a web interface to hers/his
clients, in order to perform bookings, as well as a mobile application to
manage reservations. Moreover, you can think of a driving school offering such
service to book driving lessons, let the clients pay through the system, or
sign their presence to the teaching classes, etc. Finally, any self-employed
practitioner could use the system to manage hers/his time slots, see analytics
about its most requested service, time frame, automatically collect surveys,
and communicate with hers/his clients through the notification integration.

This section needs to be revised with use cases like in UML. An example of a use case is the following:
\begin{table}[h!]
	\centering
	\begin{tabular}{|p{12,5cm}|}
		\hline
		\multicolumn{1}{|c|}{\textbf{CASO D'USO: EmissioneBiglietto}}\\
		\hline
		ID: 1\\
		\hline
		Breve descrizione:\\il Sistema emette un biglietto
valido per una specifica corsa in un determinato
giorno, previo controllo della disponibilità
del posto.\\
		\hline
		Attori primari: Impiegato\\
		\hline
		Attori secondari: None\\
		\hline
		Precondizioni:\\l'Impiegato deve essere
autenticato dal sistema.\\
		\hline
		Sequenza degli eventi:
		\vspace{-0.5\baselineskip}
		\begin{itemize}[leftmargin=1.2em]
			\item[1.] L'impiegato avvia il caso d'uso 
EmissioneBiglietto.
			\vspace{-0.5\baselineskip}
			\item[2.] L'impiegato seleziona la corsa e il giorno.
			\vspace{-0.5\baselineskip}
			\item[3.] Il Sistema mostra le corse.
			\vspace{-0.5\baselineskip}
			\item[4.] L'impiegato sceglie il tipo di biglietto da emettere.
			\vspace{-0.5\baselineskip}
			\item[] punto di estensione: ContaPosti
			\vspace{-0.5\baselineskip}
			\item[5.] Se il numero di posti disponibili è 0
			\vspace{-0.5\baselineskip}
			\begin{itemize}
			\item[5.1.] Il Sistema segnala un errore e 
	interrompe la procedura di emissione del
	biglietto.
			\end{itemize}
			\vspace{-0.8\baselineskip}
			\item[6.] Altrimenti
			\vspace{-0.5\baselineskip}
			\begin{itemize}
			\item[6.1.] Il Sistema emette il biglietto, 
	aggiornando il numero di posti 
	disponibili.
			%\vspace{-0.2\baselineskip}
			\item[6.2.] Se il biglietto è per persona
				\vspace{-0.1\baselineskip}
				\begin{itemize}[leftmargin=2.8em]
					%\vspace{-0.5\baselineskip}
					\item[6.2.1.] Il Sistema registra data e ora di emissione del biglietto
				\end{itemize}
				\item[6.3.] Se il biglietto è per autoveicolo
				\vspace{-0.1\baselineskip}
				\begin{itemize}[leftmargin=2.8em]
					%\vspace{-0.5\baselineskip}
					\item[6.3.1.] Il Sistema registra data e ora di emissione e targa del biglietto
				\end{itemize}
			\end{itemize}
			\vspace{-0.4\baselineskip}
		\end{itemize}
		\vspace{-1.5\baselineskip}
		\\
		\hline
		Postcondizioni: None\\
		\hline
		Sequenze degli eventi alternative: None\\
		\hline
	\end{tabular}
	%\caption{Utenti}
	%\label{tab:tab18}
\end{table}



\end{document}
