\documentclass{article}
\usepackage{graphicx} % Required for inserting images

\title{Prenotalo: project high-level specification.}
\author{Cecchetto A., D'Antonio S., Perella C., Roccia E., Zerpa Ruiz J. E.}
\date{January 2024}

\begin{document}

\maketitle

\section{Introduction}
Prenotalo is a simple but comprehensive booking system. The idea at its core is to have an accessible system to manage bookings and related operations.

\section{Objectives}
The aim of this project is to create a distributed system, made up of different micro-services, able to provide booking management and integration with other enriching services. The objectives of this project can be broken down into the following bullet points.
\begin{itemize}
\item a booking system
\item a user's registry
\item a payment integration
\item a notification integration
\item a survey collector
\item an analytics engine
\item a web interface
\item a mobile interface
\end{itemize}

\section{Potential users}
Such a system could potentially be used by any small business, merchant or self-employed practitioner. A booking system is useful any time there is a service offered in slots of time. Hence, some actual potential users might be lawyers, driving schools, teachers, etc, offering a service, and its clients using the system to reserve time slots of that service.

\section{Sample use cases}
To have in mind some actual use cases, you might think of lawyer offering hers/his services by appointment, providing a web interface to hers/his clients, in order to perform bookings, as well as a mobile application to manage reservations. Moreover, you can think of a driving school offering such service to book driving lessons, let the clients pay through the system, or sign their presence to the teaching classes, etc. Finally, any self-employed practitioner could use the system to manage hers/his time slots, see analytics about its most requested service, time frame, automatically collect surveys, and communicate with hers/his clients through the notification integration.

\end{document}
